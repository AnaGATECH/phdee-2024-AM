\documentclass{article}
\usepackage[utf8]{inputenc}
\usepackage{hyperref}
\hypersetup{
colorlinks=true,
    linkcolor=black,
    filecolor=black,      
    urlcolor=blue,
    citecolor=black,
}
\usepackage[letterpaper, portrait, margin=1in]{geometry}
\usepackage{enumitem}
\usepackage{amsmath}
\usepackage{booktabs}
\usepackage{graphicx}
\usepackage{titlesec}

\titleformat{\section}
{\normalfont\Large\bfseries}{\thesection}{1em}{}[{\titlerule[0.8pt]}]
  
\title{Homework 4 \\ Economics 7103}
\author{Ana Mazmishvili}
\date{February 12}
  
\begin{document}
  
\maketitle

\section{Python}

\noindent 1. See figure \ref{fig:trend}

\begin{figure}[h]
    \centering
    \includegraphics{homework 4/output/figure/trend1.pdf}
    \caption{ Bycatch by month before and after treatment. }
    \label{fig:trend}
\end{figure}

\noindent 2. See table \ref{tab:DID1}

\begin{table}[]
    \centering
    const             136310.169457
treated            11052.449649
post_treatment      2563.075526
trt_posttrt        -8956.783746
dtype: float64

    \caption{DID results}
    \label{tab:DID1}
\end{table}


\noindent 3. Each method produces quite similar results that probably only differ in rounding error:


\noindent 4. See table .  If randomization worked, the simple difference-in-means is an unbiased estimate of the treatment effect.

\section{Stata}

\noindent \textbf{1.a.} I generated firm and month indicator variables and included all of them in the regression. Stata dropped two firm indicator variables and one month indicator variable. The DID estimator is presented in the first column of the Table \ref{tab:OLS}. \\

\noindent \textbf{1.b.} After demeaning at the firm level, "firmsize", "firm" and "month" dummy variables were eliminated. The DID estimator is presented in the second column of the Table \ref{tab:OLS}. \\

\noindent \textbf{1.c} The table \ref{tab:OLS} summarizes the results of estimating models in part (a) and (b). In the first model, the results indicate that treated firms decreased bycatch by 8,085 pounds. In the demeaned OLS regression results, treated firms, on average, reduced bycatch by 8,149 pounds relative to the control group after accounting for group-specific time trends.

\begin{table}[]
    \centering
    \begin{tabular}{l*{2}{c}}
\hline\hline
                    &\multicolumn{1}{c}{(a) Original}&\multicolumn{1}{c}{(b) No FE}\\
\hline
Treatment Effect Estimates&    -8085.14&    -8149.06\\
                    &   (2619.21)&   (2489.02)\\
\hline
Method              &    With FEs&    Demeaned\\
Observations        &        1200&        1200\\
\hline\hline
\end{tabular}

    \caption{Estimating the DID estimators using OLS regression}
    \label{tab:OLS}
\end{table}

\end{document}