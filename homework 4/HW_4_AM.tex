\documentclass{article}
\usepackage[utf8]{inputenc}
\usepackage{hyperref}
\hypersetup{
colorlinks=true,
    linkcolor=black,
    filecolor=black,      
    urlcolor=blue,
    citecolor=black,
}
\usepackage[letterpaper, portrait, margin=1in]{geometry}
\usepackage{enumitem}
\usepackage{amsmath}
\usepackage{amsthm}
\usepackage{yhmath}
\usepackage{booktabs}
\usepackage{graphicx}
\usepackage{titlesec}
\usepackage[section]{placeins}
\usepackage{float}

\usepackage[flushleft]{threeparttable}
\usepackage{textcomp}
\usepackage{amssymb}
\usepackage{dsfont}


\newcommand\iid{\stackrel{\mathclap{iid}}{\sim}}
\newcommand\asym{\stackrel{\mathclap{a}}{\sim}}
\newcommand\convprob{\xrightarrow{p}}
\newcommand\convdist{\xrightarrow{d}}
\newcommand{\N}{\mathbb{N}}
\newcommand{\Z}{\mathbb{Z}}
\newcommand{\E}{\text{E}}
\newcommand{\V}{\text{Var}}
\newcommand{\Av}{\text{Avar}}
\newcommand{\se}{\text{se}}
\newcommand{\corr}{\text{Corr}}
\newcommand{\cov}{\text{Cov}}
\newcommand{\norm}{\text{Normal}}
\newcommand{\indep}{\perp \!\!\! \perp}

\titleformat{\section}
{\normalfont\Large\bfseries}{\thesection}{1em}{}[{\titlerule[0.8pt]}]
  
\title{Homework 4 \\ Economics 7103}
\author{Ana Mazmishvili}
\date{February 12}
  
\begin{document}
  
\maketitle

\section{Python}

\noindent 1. As we see from the figure \ref{fig:trend}, the treatment and control groups do not have perfect parallel trends, but the gap between them are still similar. Although the difference remains consistent over time until December 2017, a markedly sharp decrease in bycatch is observed right before treatment.

\begin{figure}[h]
    \centering
    \includegraphics{homework 4/output/figure/trend1.pdf}
    \caption{ Bycatch by months before and after treatment. }
    \label{fig:trend}
\end{figure}

\FloatBarrier

\noindent 2. I manually estimated using Python the treatment effect of the program on bycatch using the sample analog of the population difference-in-differences for the treatment and control groups in December 2017 and January 2018. The results are presented in table \ref{tab:DID1}

\begin{table}[h]
    \centering
    \begin{tabular}{rl}
\toprule
 & Sample analog of the population DID \\
\midrule
$\E [Y_{igt}|g(i) = treat, t=Dec2017] = $ & 148430.64 \\
$\E [Y_{igt}|g(i) = treat, t=Jan2018] = $ & 139612.51 \\
$\E [Y_{igt}|g(i) = control, t=Dec2017] = $ & 137228.60 \\
$\E [Y_{igt}|g(i) = control, t=Jan2018] = $ & 138001.81 \\
$\midrule DID$ =  & -9591.35 \\
\bottomrule
\end{tabular}

    \caption{The sample analog of the population DID for treatment and control groups.}
    \label{tab:DID1}
\end{table}

\FloatBarrier

\noindent \textbf{3.a. } Two-period difference-in-differences estimator is presented in the first column of Table \ref{tab:report}. The treatment effect is -9,591.31, which means that as a result of the treatment, the treated firms decrease bycatch by 9,591.31 pounds. 

\noindent \textbf{3.b.} After adding all months, the treatment effect has decreased, namely, as a result of the treatment, the treated firms decreased bycatch by 8,956.78 pounds. 

\noindent \textbf{3.c.} After controlling for other covariates, the treatment effect decreased even more and reached 8,436.28 pounds of bycatch. The outcome is much smaller than it was in equation 1. 

\noindent \textbf{3.d.} See table \ref{tab:report}. The estimator in the first column is the same what I obtained in part two, but other two estimates are smaller compared to manually calculated one. 

\begin{table}[h]
    \centering
    \input{homework 4/output/table/report}
    \caption{Three model DID estimators}
    \label{tab:report}
\end{table}


\section{Stata}

\noindent \textbf{1.a.} I generated firm and month indicator variables and included all of them in the regression. Stata dropped two firm indicator variables and one month indicator variable. The DID estimator is presented in the first column of the Table \ref{tab:OLS}. \\

\noindent \textbf{1.b.} After demeaning at the firm level, "firmsize", "firm" and "month" dummy variables were eliminated. The DID estimator is presented in the second column of the Table \ref{tab:OLS}. \\

\noindent \textbf{1.c} The Table \ref{tab:OLS} summarizes the results of estimating models in part (a) and (b). In the first model, the results indicate that treated firms decreased bycatch by 8,085 pounds. In the demeaned OLS regression results, treated firms, on average, reduced bycatch by 8,149 pounds relative to the control group after accounting for group-specific time trends.

\begin{table}[h]
    \centering
    \begin{tabular}{l*{2}{c}}
\hline\hline
                    &\multicolumn{1}{c}{(a) Original}&\multicolumn{1}{c}{(b) No FE}\\
\hline
Treatment Effect Estimates&    -8085.14&    -8149.06\\
                    &   (2619.21)&   (2489.02)\\
\hline
Method              &    With FEs&    Demeaned\\
Observations        &        1200&        1200\\
\hline\hline
\end{tabular}

    \caption{Estimating the DID estimators using OLS regression}
    \label{tab:OLS}
\end{table}

\end{document}