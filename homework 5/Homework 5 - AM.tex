\documentclass{article}
\usepackage[utf8]{inputenc}
\usepackage{hyperref}
\usepackage[letterpaper, portrait, margin=1in]{geometry}
\usepackage{enumitem}
\usepackage{amsmath}
\usepackage{booktabs}
\usepackage{graphicx}

\usepackage{titlesec}

\titleformat{\section}
{\normalfont\Large\bfseries}{\thesection}{1em}{}[{\titlerule[0.8pt]}]
  
\title{Homework 5}
\author{Economics 7103}
  
\begin{document}
  
\maketitle

\section{Python}
\noindent 1. OLS regression results are presented in the output Table \ref{tab:OLS1}. As the car becomes more fuel-efficient (able to travel more miles per gallon), its price decreases by \$22 for each additional mile per gallon. This implies that higher fuel efficiency is associated with a lower cost for the car.

\begin{table}[h]
    \centering
    \begin{center}
\begin{tabular}{lclc}
\toprule
\textbf{Dep. Variable:}    &      price       & \textbf{  R-squared:         } &     0.193   \\
\textbf{Model:}            &       OLS        & \textbf{  Adj. R-squared:    } &     0.191   \\
\textbf{Method:}           &  Least Squares   & \textbf{  F-statistic:       } &     119.1   \\
\textbf{Date:}             & Mon, 19 Feb 2024 & \textbf{  Prob (F-statistic):} &  4.09e-47   \\
\textbf{Time:}             &     07:14:25     & \textbf{  Log-Likelihood:    } &   -9574.6   \\
\textbf{No. Observations:} &        1000      & \textbf{  AIC:               } & 1.916e+04   \\
\textbf{Df Residuals:}     &         997      & \textbf{  BIC:               } & 1.917e+04   \\
\textbf{Df Model:}         &           2      & \textbf{                     } &             \\
\textbf{Covariance Type:}  &    nonrobust     & \textbf{                     } &             \\
\bottomrule
\end{tabular}
\begin{tabular}{lcccccc}
               & \textbf{coef} & \textbf{std err} & \textbf{t} & \textbf{P$> |$t$|$} & \textbf{[0.025} & \textbf{0.975]}  \\
\midrule
\textbf{mpg}   &     -22.2121  &       16.521     &    -1.344  &         0.179        &      -54.632    &       10.208     \\
\textbf{car}   &   -3202.5704  &      261.823     &   -12.232  &         0.000        &    -3716.357    &    -2688.783     \\
\textbf{const} &    2.248e+04  &      490.096     &    45.874  &         0.000        &     2.15e+04    &     2.34e+04     \\
\bottomrule
\end{tabular}
\begin{tabular}{lclc}
\textbf{Omnibus:}       &  1.020 & \textbf{  Durbin-Watson:     } &    1.989  \\
\textbf{Prob(Omnibus):} &  0.601 & \textbf{  Jarque-Bera (JB):  } &    0.888  \\
\textbf{Skew:}          &  0.039 & \textbf{  Prob(JB):          } &    0.642  \\
\textbf{Kurtosis:}      &  3.123 & \textbf{  Cond. No.          } &     151.  \\
\bottomrule
\end{tabular}
%\caption{OLS Regression Results}
\end{center}

Notes: \newline
 [1] Standard Errors assume that the covariance matrix of the errors is correctly specified.
    \caption{OLS Regression results}
    \label{tab:OLS1}
\end{table}

\noindent 2. In this model, I am concerned about the \textbf{Omitted Variable Bias}. For instance, car characteristics other than being a Sedan or SUV, are omitted from the model that might be correlated with both the price and mpg. This can lead to biased and inconsistent estimates. 

\clearpage

\noindent 3. (a-c) Manully calculated 2Stage-least-square results are presented in the output Table \ref{tab:3OLS}. In the first column is presented regression results from part 3a, in the second column 3b and in the third one 3c. 

\begin{table}[h]
    \centering
    \begin{table}[!htbp] \centering
\begin{tabular}{@{\extracolsep{5pt}}lccc}
\\[-1.8ex]\hline
\hline \\[-1.8ex]
& \multicolumn{3}{c}{\textit{Dependent variable: price}} \
\cr \cline{2-4}
\\[-1.8ex] & (1) & (2) & (3) \\
\hline \\[-1.8ex]
 MPG & 150.43$^{**}$ & 157.06$^{**}$ & 10165.74$^{}$ \\
& (62.16) & (62.02) & (26559.83) \\
 Car & -4676.09$^{***}$ & -4732.67$^{***}$ & -90156.39$^{}$ \\
& (574.37) & (573.29) & (226687.35) \\
\hline \\[-1.8ex]
 F-statistics from the 1st Stage & 256.8 & 257.02 & 203.66 \\
 Observations & 1000 & 1000 & 1000 \\
 $R^2$ & 0.20 & 0.20 & 0.19 \\
 Adjusted $R^2$ & 0.19 & 0.19 & 0.19 \\
 Residual Std. Error & 3481.08 & 3480.12 & 3491.04 \\
 F Statistic & 121.62$^{***}$ & 121.97$^{***}$ & 118.09$^{***}$ \\
\hline
\hline \\[-1.8ex]
\textit{Note:} & \multicolumn{3}{r}{$^{*}$p$<$0.1; $^{**}$p$<$0.05; $^{***}$p$<$0.01} \\
\end{tabular}
\end{table}
    \caption{Manually Calculated IV Results}
    \label{tab:3OLS}
\end{table}

\noindent 3.d. The instrumental variables should satisfy the following requirements: 
1. The instrument must clearly affect the endogenous variable, in this case, 'mpg'. F-statistics from the first stage 
2. \textbf{Exclusion restriction} - instrumental variable should be uncorrelated with any other determinants of the dependent variables. In other words, an instrument must have a unique channel for causal effect. In this case, instruments (weight, weight square, and height) should not be correlated with the variable 'car'. The only channel impacting the prices should be through 'mpg'.

In words, what are the different exclusion restrictions required for parts (a)-(c)? Does this seem reasonable for these instruments?


\noindent 3.e Compare and contrast the estimated coefficient on mpg from parts (a)-(c). What explains the discrepancies?

\noindent 4. Calculate the IV estimate using GMM with weight as the excluded instrument. (Look for the Linearmodels function IVGMM). Report the estimated second-stage coefficient and standard error or confidence interval for mpg What factors account for the differences in the standard errors?

IV GMM results are presented in the Table \ref{tab:GMM}

\begin{table}[h]
    \centering
    \begin{tabular}{lcc}
\toprule
 & IVGMM \\
\midrule
Miles per gallon & 150.43 \\
  & (63.05) \\
=1 if the vehicle is sedan & -4676.09 \\
  & (589.70) \\
\bottomrule
\end{tabular}

    \caption{IV estimate calculated using GMM}
    \label{tab:GMM}
\end{table}

\clearpage

\section{Stata}

\noindent 1. The limited information maximum likelihood estimates are presented in the Table \ref{tab:stataIV}

\begin{table}[h]
    \centering
    \begin{tabular}{lc} \hline
 & (1) \\
VARIABLES & The limited information maximum likelihood estimate \\ \hline
 &  \\
Miles per gallon & 150.4** \\
 & (63.05) \\
=1 if the vehicle is a sedan & -4,676*** \\
 & (589.7) \\
Constant & 17,628*** \\
 & (1,773) \\
 &  \\
Observations & 1,000 \\
 R-squared & 0.104 \\ \hline
\multicolumn{2}{c}{ Robust standard errors in parentheses} \\
\multicolumn{2}{c}{ *** p$<$0.01, ** p$<$0.05, * p$<$0.1} \\
\end{tabular}

    \caption{IV estimate calculated using GMM}
    \label{tab:stataIV}
\end{table}


\noindent 2. Use weakivtest to estimate the Montiel-Olea-Pflueger effective F-statistic. What is the 5\% critical value, the F-statistic, and conclusion?


\end{document}