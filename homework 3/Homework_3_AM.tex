\documentclass{article}
\usepackage[utf8]{inputenc}
\usepackage{hyperref}
\usepackage[letterpaper, portrait, margin=1in]{geometry}
\usepackage{enumitem}
\usepackage{amsmath}
\usepackage{booktabs}
\usepackage{graphicx}
\usepackage{hyperref}
\usepackage{titlesec}
\usepackage[section]{placeins}

\hypersetup{
colorlinks=true,
    linkcolor=black,
    filecolor=black,      
    urlcolor=blue,
    citecolor=black,
}

  
\title{Economics 7103 - Homework 3}
\author{Ana Mazmishvili}
\date{ February 3, 2024 }
  
\begin{document}
  
\maketitle 
\section{Description of the problem}
Suppose that for a home \(i\), you think the underlying relationship between electricity use and predictor
variables is \(y_i= e^{\alpha} \delta^{d_i} z_i^\gamma e^{\eta_i}\) where 
\begin{itemize}
    \item \(e\) is Euler’s number or the base of the natural logarithm 
    \item \(d_i\) is a binary variable equal to one if home \(i\) received the retrofit program
    \item \(z_i\) is a vector of the other control variables
    \item \(\eta_i\) is unobserved error 
    \item \({\alpha, \delta, \gamma}\) are parameters to estimate.
\end{itemize}

\subsection*{Question 1.a} 
Show that \(ln(y_i)= \alpha + ln(\delta)d_i + \gamma ln(z_i) + \eta_i\)

\textbf{Response:} 
The given equation, \(y_i= e^{\alpha} \delta^{d_i} z_i^\gamma e^{\eta_i}\), represents a power regression model, which is a non-linear regression model. To transform it into a linear regression model, we need to take the natural log of both sides of the equation. We call this log-log transformation. I will use the properties of logarithms and exponential functions in this transformation:
\begin{itemize}
    \item \textbf{Taking log:} \(ln(y_i)= ln(e^{\alpha} \delta^{d_i} z_i^\gamma e^{\eta_i})\)
    \item \textbf{Product Rule:} \(ln(y_i)= ln(e^{\alpha})+ln(\delta^{d_i}) + ln(z_i^\gamma)+ln(e^{\eta_i})\)
    \item \textbf{Power Rule:} \(ln(y_i)= \alpha ln(e)+ln(\delta)d_i + \gamma ln(z_i) +\eta_iln(e)\)
    \item \textbf{The natural logarithm of \(e\) = 1:} \(ln(y_i)= \alpha + ln(\delta)d_i + \gamma ln(z_i) + \eta_i\)      
\end{itemize}





\subsection*{Question 1.b} What is the intuitive interpretation of \(\delta\) ? 

\textbf{Response:} 
\(\delta\)


\subsection*{Question 1.c} Show that\(\frac{\triangle y_i}{\triangle d_i} = \frac{\delta -1}{\delta^{d_i}} y_i \) . What is the intuitive interpretation of \(\frac{\triangle y_i}{\triangle d_i} \).

\textbf{Response:} 

\subsection*{Question 1.d} Show that \(\frac{\partial y_i}{\partial z_i} = \gamma \frac{y_i}{z_i}\). What is the intuitive interpretation of \(\frac{\partial y_i}{\partial z_i}\) when \(z_i\) is the size of the home in square feet?

\textbf{Response:} 
\[\frac{\partial y_i}{\partial z_i} = \frac{\partial (e^{\alpha} \delta^{d_i} z_i^\gamma e^{\eta_i})}{\partial z_i} = \\ 
                                       e^{\alpha} \delta^{d_i} \gamma z_i^{\gamma - 1} e^{\eta_i} = \\
                                      e^{\alpha} \delta^{d_i} z_i^\gamma e^{\eta_i} \gamma z_i^{- 1} = \\ 
                                      \gamma \frac{y_i}{z_i} \]
\(\frac{\partial y_i}{\partial z_i}\) describes the marginal change in the electricity consumption of HHs when the size of the home in square feet changes. 

\subsection*{Question 1.e}
\textbf{Response:} 

\begin{table}[hbt!]
    \centering
    \begin{tabular}{lccc}
\toprule
 & Parameter estimates & AME estimates \\
\midrule
Constant & -0.769000 &   \\
  & [-1.93, 0.315] &   \\
=1 if home received retrofit & 0.904000 & -113.975000 \\
  & [0.894, 0.915] & [-127.401, -99.894] \\
Square feet of home & 0.894000 & 0.629000 \\
  & [0.88, 0.908] & [0.617, 0.64] \\
Outdoor average temperature (\textdegree F) & 0.281000 & 3.997000 \\
  & [0.048, 0.541] & [0.682, 7.759] \\
Observations & 1000 & 1000 \\
\bottomrule
\end{tabular}

    \caption{Estimated parameters and AME (Python)}
    \label{tab:parameters}
        \end{table}


\subsection*{Question 1.f}
\textbf{Response:} 



\end{document}