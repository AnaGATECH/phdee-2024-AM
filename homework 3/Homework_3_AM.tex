\documentclass{article}
\usepackage[utf8]{inputenc}
\usepackage{hyperref}
\usepackage[letterpaper, portrait, margin=1in]{geometry}
\usepackage{enumitem}
\usepackage{amsmath}
\usepackage{booktabs}
\usepackage{graphicx}
\usepackage{hyperref}
\usepackage{titlesec}
\usepackage[section]{placeins}

\hypersetup{
colorlinks=true,
    linkcolor=black,
    filecolor=black,      
    urlcolor=blue,
    citecolor=black,
}

  
\title{Economics 7103 - Homework 3}
\author{Ana Mazmishvili}
\date{ February 3, 2024 }
  
\begin{document}
  
\maketitle 
\section{Stata or Python}
Suppose that for a home \(i\), you think the underlying relationship between electricity use and predictor
variables is \(y_i= e^{\alpha} \delta^{d_i} z_i^\gamma e^{\eta_i}\) where 
\begin{itemize}
    \item \(e\) is Euler’s number or the base of the natural logarithm 
    \item \(d_i\) is a binary variable equal to one if home \(i\) received the retrofit program
    \item \(z_i\) is a vector of the other control variables
    \item \(\eta_i\) is unobserved error 
    \item \({\alpha, \delta, \gamma}\) are parameters to estimate.
\end{itemize}

\subsection*{Question 1.a} 
Show that \(ln(y_i)= \alpha + ln(\delta)d_i + \gamma ln(z_i) + \eta_i\)

\textbf{Response:} 
The given equation, \(y_i= e^{\alpha} \delta^{d_i} z_i^\gamma e^{\eta_i}\), represents a power regression model, which is a non-linear regression model. To transform it into a linear regression model, we need to take the natural log of both sides of the equation. We call this log-log transformation. I will use the properties of logarithms and exponential functions in this transformation:
\begin{itemize}
    \item \textbf{Taking log:} \(ln(y_i)= ln(e^{\alpha} \delta^{d_i} z_i^\gamma e^{\eta_i})\)
    \item \textbf{Product Rule:} \(ln(y_i)= ln(e^{\alpha})+ln(\delta^{d_i}) + ln(z_i^\gamma)+ln(e^{\eta_i})\)
    \item \textbf{Power Rule:} \(ln(y_i)= \alpha ln(e)+ln(\delta)d_i + \gamma ln(z_i) +\eta_iln(e)\)
    \item \textbf{The natural logarithm of \(e\) = 1:} \(ln(y_i)= \alpha + ln(\delta)d_i + \gamma ln(z_i) + \eta_i\)      
\end{itemize}





\subsection*{Question 1.b} What is the intuitive interpretation of \(\delta\) ? 

\textbf{Response:} 
\(\delta\)


\subsection*{Question 1.c}
\textbf{Response:} 


\[\hat{\beta} = (X^{'}X)^{-1}X^{'}Y \]

 

\subsection*{Question 1.d}
\textbf{Response:} 


\subsection*{Question 1.e}
\textbf{Response:} 


\subsection*{Question 1.f}
\textbf{Response:} 



\end{document}